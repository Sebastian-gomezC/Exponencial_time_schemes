\documentclass[12pt]{article}
\usepackage[utf8x]{inputenc}
\usepackage[spanish]{babel}
\usepackage{url}
\usepackage{cancel}
\usepackage{lipsum}
\usepackage{graphicx}
\usepackage{color}
\usepackage{float}
\usepackage{amsmath}
\usepackage{listings}
\usepackage{amssymb}
\usepackage{listings}
\usepackage{geometry}
\usepackage{pythonhighlight}
\usepackage{tikz}
\geometry{
	letterpaper,
	total={170mm,240mm},
	left=20mm,
	top=20mm,
}

\title{
	\vspace{-25mm}\begin{figure}[h]
		\centering
		\includegraphics[width=1in]{Escudo_de_la_Universidad_Nacional_de_Colombia.png}
		\label{escudo}
	\end{figure}
	Aplicaciones de elementos finitos\\
	\Large Universidad nacional de Colombia\\ 
	Facultad de ingeniería }

\author{Jhon Sebastian Gómez Castillo -
 \textit{jsgomezca@unal.edu.co}
}

\begin{document}
	
\section{Esquemas de integración temporal}

\subsection{esquemas bac}

\section{Casos de prueba}
Se eligieron dos casos como Benchmark para probar el desarrollo computacional implementado. el primero es 
	\maketitle
\begin{figure}[H]
	\centering
	\label{dominio del problema}
	\caption{dominio del problema}
	\begin{tikzpicture}[scale=2]
		
		% Dibujar el cuadrado
		\draw (0,0) --node[below]{T=0} (3,0) -- node[right]{T=0}  (3,3) --node[above]{q=0}  (0,3) -- node[left]{q=0} cycle;
		
		
		% Fuente en el interior
		\filldraw[fill=blue!20,draw=blue,thick] (1.5,1.5) circle [radius=0.07] node[right,black] {Q};
	\end{tikzpicture}
\end{figure}


	
\end{document}